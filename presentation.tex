\documentclass[aspectratio=169]{beamer}
\usepackage[utf8]{inputenc}
\usepackage[spanish]{babel}
\usepackage{graphicx}
\usepackage{amsmath}
\usepackage{amsfonts}
\usepackage{amssymb}
\usepackage{booktabs}
\usepackage{xcolor}
\usepackage{tikz}
\usepackage{pgfplots}
\pgfplotsset{compat=1.17}

% Tema y colores
\usetheme{Madrid}
\usecolortheme{default}
\setbeamercolor{structure}{fg=blue!80!black}
\setbeamercolor{title}{fg=white}
\setbeamercolor{frametitle}{fg=white}
\setbeamercolor{block title}{bg=blue!80!black,fg=white}
\setbeamercolor{block body}{bg=white,fg=black}
\setbeamercolor{block title alerted}{bg=red!80!black,fg=white}
\setbeamercolor{block body alerted}{bg=white,fg=black}
\setbeamercolor{block title example}{bg=green!70!black,fg=white}
\setbeamercolor{block body example}{bg=white,fg=black}

% Remover sombreados que causan problemas de visualización
\setbeamertemplate{blocks}[default]
\setbeamertemplate{navigation symbols}{}
\setbeamertemplate{footline}[frame number]

% Configuración de fuentes
\setbeamerfont{title}{size=\Large,series=\bfseries}
\setbeamerfont{frametitle}{size=\large,series=\bfseries}

% Información del documento
\title[TTAC TestDataScience]{Proyectos de Machine Learning y Series Temporales}
\subtitle{Análisis Completo de Datasets UCI}
\author{Tomás Travis Alonso Cremnitz}
\date{Septiembre 2025}

\begin{document}

% Diapositiva de título
\begin{frame}
    \titlepage
\end{frame}

% Índice
\begin{frame}{Contenido}
    \tableofcontents
\end{frame}

\section{Introducción y Objetivos}

\begin{frame}{Planteamiento del Caso}
    \begin{block}{Objetivo General}
        Desarrollar e implementar dos sistemas completos de Machine Learning utilizando datasets reales de UCI ML Repository
    \end{block}
    
    \vspace{0.5cm}
    
    \begin{columns}[T]
        \begin{column}{0.48\textwidth}
            \begin{alertblock}{TEST 1: Clasificación}
                \begin{itemize}
                    \item \textbf{Dataset}: Wine Quality UCI
                    \item \textbf{Muestras}: 6,497 vinos
                    \item \textbf{Objetivo}: Clasificar calidad (3-9)
                    \item \textbf{Modelos}: RF, SVM, XGBoost
                \end{itemize}
            \end{alertblock}
        \end{column}
        
        \begin{column}{0.48\textwidth}
            \begin{alertblock}{TEST 2: Series Temporales}
                \begin{itemize}
                    \item \textbf{Dataset}: Air Quality UCI
                    \item \textbf{Muestras}: 9,358 registros horarios
                    \item \textbf{Objetivo}: Forecasting CO(GT)
                    \item \textbf{Modelos}: ARIMA, LSTM, Prophet
                \end{itemize}
            \end{alertblock}
        \end{column}
    \end{columns}
\end{frame}

\begin{frame}{Metodología de Trabajo}
    \begin{enumerate}
        \item \textbf{Análisis Exploratorio de Datos (EDA)}
        \begin{itemize}
            \item Carga y validación de datasets UCI reales
            \item Análisis estadístico descriptivo completo
            \item Visualización de distribuciones y correlaciones
        \end{itemize}
        
        \item \textbf{Preprocesamiento de Datos}
        \begin{itemize}
            \item Limpieza y tratamiento de valores faltantes
            \item Escalado de características
            \item División estratificada train/test
        \end{itemize}
        
        \item \textbf{Modelado y Evaluación}
        \begin{itemize}
            \item Implementación de múltiples algoritmos
            \item Validación cruzada y métricas robustas
            \item Comparación de rendimiento
        \end{itemize}
    \end{enumerate}
\end{frame}

\section{TEST 1: Clasificación de Calidad de Vinos}

\begin{frame}{Dataset UCI Wine Quality}
    \begin{columns}[T]
        \begin{column}{0.6\textwidth}
            \begin{block}{Características del Dataset}
                \begin{itemize}
                    \item \textbf{Fuente}: UCI ML Repository
                    \item \textbf{Total}: 6,497 muestras reales
                    \item \textbf{Vinos tintos}: 1,599 (24.6\%)
                    \item \textbf{Vinos blancos}: 4,898 (75.4\%)
                    \item \textbf{Features}: 11 propiedades fisicoquímicas
                    \item \textbf{Target}: Calidad (escala 3-9)
                \end{itemize}
            \end{block}
        \end{column}
        
        \begin{column}{0.35\textwidth}
            \begin{table}
                \centering
                \small
                \begin{tabular}{cc}
                    \toprule
                    \textbf{Calidad} & \textbf{Muestras} \\
                    \midrule
                    3 & 30 \\
                    4 & 216 \\
                    5 & 2,138 \\
                    6 & 2,836 \\
                    7 & 1,079 \\
                    8 & 193 \\
                    9 & 5 \\
                    \bottomrule
                \end{tabular}
                \caption{Distribución de calidades}
            \end{table}
        \end{column}
    \end{columns}
    
    \vspace{0.3cm}
    
    \begin{exampleblock}{Principales Hallazgos del EDA}
        \begin{itemize}
            \item Distribución centrada en calidades medias (5-7)
            \item Desbalance significativo en clases extremas (3, 9)
            \item Diferencias químicas marcadas entre vinos tintos y blancos
        \end{itemize}
    \end{exampleblock}
\end{frame}

\begin{frame}{Modelos de Clasificación Implementados}
    \begin{table}
        \centering
        \begin{tabular}{lccc}
            \toprule
            \textbf{Modelo} & \textbf{Accuracy} & \textbf{F1-Score} & \textbf{Tiempo} \\
            \midrule
            Random Forest & \textbf{69.31\%} & \textbf{68.30\%} & $\sim$2s \\
            SVM & 65.84\% & 64.12\% & $\sim$8s \\
            XGBoost & 67.92\% & 66.55\% & $\sim$5s \\
            \bottomrule
        \end{tabular}
        \caption{Resultados de clasificación en Wine Quality UCI}
    \end{table}
    
    \vspace{0.5cm}
    
    \begin{columns}[T]
        \begin{column}{0.48\textwidth}
            \begin{block}{Random Forest (Mejor Modelo)}
                \begin{itemize}
                    \item \textbf{Hiperparámetros}: 100 árboles, max\_depth=10
                    \item \textbf{Ventajas}: Robusto, interpretable
                    \item \textbf{Feature Importance}: Alcohol, volatil acidity principales
                \end{itemize}
            \end{block}
        \end{column}
        
        \begin{column}{0.48\textwidth}
            \begin{alertblock}{Métricas de Validación}
                \begin{itemize}
                    \item \textbf{Validación estratificada}: 5-fold CV
                    \item \textbf{F1-Score weighted}: Apropiado para desbalance
                    \item \textbf{Matriz de confusión}: Análisis por clase
                \end{itemize}
            \end{alertblock}
        \end{column}
    \end{columns}
\end{frame}

\section{TEST 2: Forecasting de Series Temporales}

\begin{frame}{Dataset UCI Air Quality}
    \begin{columns}[T]
        \begin{column}{0.6\textwidth}
            \begin{block}{Características del Dataset}
                \begin{itemize}
                    \item \textbf{Fuente}: UCI ML Repository
                    \item \textbf{Total}: 9,358 registros horarios
                    \item \textbf{Período}: Mar 2004 - Feb 2005
                    \item \textbf{Variables}: 15 sensores ambientales
                    \item \textbf{Target}: CO(GT) - Monóxido de Carbono
                    \item \textbf{Frecuencia}: Mediciones horarias
                \end{itemize}
            \end{block}
            
            \begin{exampleblock}{Compliance TEST 2}
                \begin{itemize}
                    \item \textcolor{green}{\checkmark} Dataset NO financiero
                    \item \textcolor{green}{\checkmark} Variable NO estacional
                    \item \textcolor{green}{\checkmark} Forecasting 100 períodos
                \end{itemize}
            \end{exampleblock}
        \end{column}
        
        \begin{column}{0.35\textwidth}
            \begin{table}
                \centering
                \small
                \begin{tabular}{ll}
                    \toprule
                    \textbf{Métrica} & \textbf{Valor} \\
                    \midrule
                    Media CO(GT) & 2.11 mg/m³ \\
                    Std CO(GT) & 1.87 mg/m³ \\
                    Min CO(GT) & -200 mg/m³ \\
                    Max CO(GT) & 11.9 mg/m³ \\
                    Valores válidos & 89.4\% \\
                    \bottomrule
                \end{tabular}
                \caption{Estadísticas CO(GT)}
            \end{table}
        \end{column}
    \end{columns}
\end{frame}

\begin{frame}{Modelos de Forecasting Implementados}
    \begin{table}
        \centering
        \begin{tabular}{lcccc}
            \toprule
            \textbf{Modelo} & \textbf{MAE} & \textbf{RMSE} & \textbf{MAPE} & \textbf{Tiempo} \\
            \midrule
            ARIMA(1,1,1) & \textbf{0.3860} & \textbf{0.4651} & \textbf{21.84\%} & $\sim$2s \\
            LSTM & En desarrollo & En desarrollo & TBD & $\sim$15s \\
            Prophet & En desarrollo & En desarrollo & TBD & $\sim$3s \\
            \bottomrule
        \end{tabular}
        \caption{Resultados de forecasting en Air Quality UCI (100 períodos)}
    \end{table}
    
    \vspace{0.5cm}
    
    \begin{columns}[T]
        \begin{column}{0.48\textwidth}
            \begin{block}{ARIMA (Modelo Principal)}
                \begin{itemize}
                    \item \textbf{Configuración}: AR(1), I(1), MA(1)
                    \item \textbf{Estacionariedad}: Validada con ADF test
                    \item \textbf{Intervalos confianza}: 95\% implementados
                \end{itemize}
            \end{block}
        \end{column}
        
        \begin{column}{0.48\textwidth}
            \begin{alertblock}{Performance Destacada}
                \begin{itemize}
                    \item \textbf{MAPE 21.84\%}: Excelente para datos ambientales
                    \item \textbf{Horizonte 100}: Cumple requisitos TEST 2
                    \item \textbf{Reproducible}: Scripts CLI automatizados
                \end{itemize}
            \end{alertblock}
        \end{column}
    \end{columns}
\end{frame}

\section{Arquitectura y Metodología}

\begin{frame}{Arquitectura de los Proyectos}
    \begin{columns}[T]
        \begin{column}{0.48\textwidth}
            \begin{block}{Estructura Modular}
                \begin{itemize}
                    \item \textbf{src/}: Código fuente organizado en paquetes
                    \item \textbf{notebooks/}: Análisis interactivos (EDA + Modeling)
                    \item \textbf{tests/}: Framework de testing con pytest
                    \item \textbf{data/}: Pipeline raw → processed → final
                \end{itemize}
            \end{block}
        \end{column}
        
        \begin{column}{0.48\textwidth}
            \begin{exampleblock}{Herramientas CLI}
                \textbf{TEST 1 - Clasificación:}
                \begin{itemize}
                    \item \texttt{train\_model.py}: Entrenamiento
                    \item \texttt{inference.py}: Predicciones
                \end{itemize}
                
                \textbf{TEST 2 - Series Temporales:}
                \begin{itemize}
                    \item \texttt{train\_model.py}: Múltiples modelos
                    \item \texttt{forecast.py}: Predicción 100 períodos
                \end{itemize}
            \end{exampleblock}
            
            \begin{block}{Stack Tecnológico}
                \begin{itemize}
                    \item \textbf{Core}: pandas, scikit-learn, statsmodels
                    \item \textbf{Deep Learning}: TensorFlow/Keras
                    \item \textbf{Viz}: matplotlib, seaborn, plotly
                \end{itemize}
            \end{block}
        \end{column}
    \end{columns}
\end{frame}

\section{Resultados y Análisis}

\begin{frame}{Comparación de Resultados}
    \begin{table}
        \centering
        \begin{tabular}{lllll}
            \toprule
            \textbf{Proyecto} & \textbf{Mejor Modelo} & \textbf{Métrica Principal} & \textbf{Dataset} & \textbf{Compliance} \\
            \midrule
            TEST 1 & Random Forest & 69.31\% Accuracy & Wine Quality (6,497) & \textcolor{green}{\checkmark} UCI Real \\
            TEST 2 & ARIMA & 21.84\% MAPE & Air Quality (9,358) & \textcolor{green}{\checkmark} No financiero \\
            \bottomrule
        \end{tabular}
    \end{table}
    \begin{columns}[T]
        \begin{column}{0.48\textwidth}
            \begin{block}{TEST 1: Clasificación}
                \textbf{Fortalezas:}
                \begin{itemize}
                    \item Accuracy superior al 65\% baseline
                    \item F1-Score balanceado para clases desbalanceadas
                    \item Feature importance interpretable
                \end{itemize}
                
                \textbf{Desafíos:}
                \begin{itemize}
                    \item Desbalance en clases extremas (3, 9)
                    \item Variabilidad en evaluación humana
                \end{itemize}
            \end{block}
        \end{column}
        
        \begin{column}{0.48\textwidth}
            \begin{block}{TEST 2: Forecasting}
                \textbf{Fortalezas:}
                \begin{itemize}
                    \item MAPE < 25\% para datos ambientales
                    \item Horizonte 100 períodos cumplido
                    \item Intervalos de confianza implementados
                \end{itemize}
                
                \textbf{Desafíos:}
                \begin{itemize}
                    \item Valores faltantes en dataset original
                    \item Optimización de LSTM en progreso
                \end{itemize}
            \end{block}
        \end{column}
    \end{columns}
\end{frame}

\begin{frame}{Análisis de Features y Variables}
    \begin{columns}[T]
        \begin{column}{0.48\textwidth}
            \begin{block}{Wine Quality - Feature Importance}
                \textbf{Top 5 Variables (Random Forest):}
                \begin{enumerate}
                    \item \textbf{alcohol}: 18.3\%
                    \item \textbf{volatile acidity}: 14.7\%
                    \item \textbf{sulphates}: 12.1\%
                    \item \textbf{total sulfur dioxide}: 10.8\%
                    \item \textbf{density}: 9.4\%
                \end{enumerate}
                
                \vspace{0.3cm}
                \textbf{Insight:} El alcohol es el predictor más fuerte de calidad, seguido de propiedades relacionadas con acidez y conservantes.
            \end{block}
        \end{column}
        
        \begin{column}{0.48\textwidth}
            \begin{block}{Air Quality - Análisis Temporal}
                \textbf{Características CO(GT):}
                \begin{itemize}
                    \item \textbf{Proceso estacionario}: ADF test p < 0.01
                    \item \textbf{Autocorrelación}: AR(1) coef = 0.2
                    \item \textbf{Estacionalidad}: No significativa
                    \item \textbf{Tendencia}: Ligeramente decreciente
                \end{itemize}
                
                \vspace{0.3cm}
                \textbf{Insight:} Variable ideal para ARIMA, sin patrones estacionales complejos que requieran models más sofisticados.
            \end{block}
        \end{column}
    \end{columns}
\end{frame}

\section{Conclusiones y Trabajo Futuro}

\begin{frame}{Trabajo Futuro y Mejoras}
    \begin{columns}[T]
        \begin{column}{0.48\textwidth}
            \begin{block}{Mejoras Inmediatas}
                \textbf{TEST 1:}
                \begin{itemize}
                    \item Ensemble methods (RF + XGBoost)
                    \item SMOTE para balancear clases extremas
                    \item Feature engineering: ratios químicos
                \end{itemize}
                
                \textbf{TEST 2:}
                \begin{itemize}
                    \item Optimización completa de LSTM
                    \item Prophet fine-tuning para datos ambientales
                    \item Ensemble ARIMA + LSTM
                \end{itemize}
            \end{block}
        \end{column}
        
        \begin{column}{0.48\textwidth}
            \begin{block}{Extensiones Avanzadas}
                \begin{itemize}
                    \item \textbf{MLOps}: CI/CD con GitHub Actions
                    \item \textbf{API REST}: FastAPI para serving
                    \item \textbf{Dashboard}: Streamlit interactivo
                    \item \textbf{AutoML}: Hyperparameter optimization
                    \item \textbf{Monitoring}: Model drift detection
                \end{itemize}
            \end{block}
        \end{column}
    \end{columns}
\end{frame}

\begin{frame}[plain]
    \begin{center}
        \vspace{2cm}
        {\Huge \textbf{Gracias}}
        
        \vspace{1cm}
        
        
        \vspace{1.5cm}
        
        
        \vspace{0.5cm}
        
    \end{center}
\end{frame}

\end{document}